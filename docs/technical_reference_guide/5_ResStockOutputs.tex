\chapter{ResStock Outputs}\label{sec:resstock_outputs}
ResStock produces a range of results around energy, housing characteristics, schedules, emissions, and costs. This section overviews the outputs available in the latest ResStock data release (2024 release 2). The data dictionary that summarizes these outputs is available with the \href{https://oedi-data-lake.s3.amazonaws.com/nrel-pds-building-stock/end-use-load-profiles-for-us-building-stock/2024/resstock_amy2018_release_2/data_dictionary.tsv}{data release}.  

\section{Building Characteristics}
%Tables - basically our data dictionaries (ideally automated). Data dictionaries are already automated but need heavy QC
The ResStock workflow generates a sample of residential housing units using the conditional distributions and sampling methodology described in Section \ref{sec:sampling_methodology}. Each housing unit sample is specified using a set of characteristics, which include location, housing type, vintage,  heating fuel, building size and geometry, materials, information on the HVAC system, insulation, infiltration, appliances, as well as information on occupant demographics and  behavior. These characteristics are used for the model creation process, but they are also available as metadata associated with the results. The metadata is useful for performing analysis and slicing the data into segments. Some characteristics directly impact the OpenStudio model creation (e.g., wall insulation levels), whereas others are meta-parameters used as tags (e.g., ISO/RTO region) or as characteristics that correlate with other energy-relevant characteristics (e.g., vintage). Each of these characteristics is described in detail in Section \ref{sec:resstock_inputs}, and are summarized in Table \ref{table:hc_output_fields}. In published ResStock datasets, these housing characteristics are the fields that use the ``in'' prefix followed by the housing characteristic name, for example ``in.windows'' for the Windows characteristic. If an upgrade or measure changes a housing characteristic, the post-upgrade characteristic uses the ``upgrade'' prefix, for example ``upgrade.windows.'' 

Most national ResStock data releases have approximately one sample for every 250 housing units that actually exist in the modeled geography, though this can vary if the sample size is increased or decreased. Taken together, the set of samples describes the housing stock with its intrinsic variety and is used as both an output of ResStock and an input to further steps in the ResStock workflow (see Section \ref{sec:workflow} for more information about the workflow). 

Note that ResStock assigns certain characteristics that are not always used. Two examples of this are \textit{cooling setpoint} in housing units with no cooling system, and \textit{clothes washer usage level} in housing units with no clothes washer. This doesn't impact the building energy models, this is merely a mechanism that allows for variability in probability distributions if the schedules are used for an upgrade. These characteristics, however, can lead to confusion because they're preserved in the metadata as being assigned even if they're not modeled. For further discussion see the the BuildStock Batch \href{https://buildstockbatch.readthedocs.io/en/v2023.10.0/project_defn.html#upgrade-scenarios}{Upgrade Scenario} documentation.


%%TODO - updated to 2024.2

\begin{customLongTable}{ |p{6cm}|p{10.0cm}| }
{ResStock building characteristic output field names and descriptions} {table:hc_output_fields} 
{\textbf{Field Name} & \textbf{Field Description}} 
        in.ahs\_region & American Housing Survey region \\ \hline
        in.aiannh\_area & The sample is or is not located in census-designated American Indian/Alaska Native/Native Hawaiian Area \\ \hline
        in.area\_median\_income & Area median income of the household occupying the housing unit \\ \hline
        in.ashrae\_iecc\_climate\_zone\_2004 & IECC climate zone according to ASHRAE 169 in 2004 and IECC in 2012 \\ \hline
        in.ashrae\_iecc\_climate\_zone\_2004\_2\_a\_split & Climate zone according to ASHRAE 169 in 2004 and IECC in 2012, where climate zone 2A is split between counties in TX, LA versus FL, GA, AL, and MS \\ \hline
        in.bathroom\_spot\_vent\_hour & Bathroom spot ventilation daily start hour \\ \hline
        in.battery & The presence, size, location, and efficiency of an on-site battery (not modeled in 2024.2 dataset) \\ \hline
        in.bedrooms & Number of bedrooms \\ \hline
        in.building\_america\_climate\_zone & Building America climate zone \\ \hline
        in.cec\_climate\_zone & California Energy Code climate zone \\ \hline
        in.ceiling\_fan & Presence and energy usage of ceiling fans at medium speed \\ \hline
        in.census\_division & 2010 U.S. Census Division \\ \hline
        in.census\_division\_recs & Census Division as used in RECS 2015 \\ \hline
        in.census\_region & 2010 U.S. Census Region \\ \hline
        in.city & The census-designated city where the sample is located \\ \hline
        in.clothes\_dryer & The presence, rated efficiency, and fuel type of the clothes dryer in the housing unit \\ \hline
        in.clothes\_dryer\_usage\_level & Clothes dryer energy usage level multiplier \\ \hline
        in.clothes\_washer & Presence and rated efficiency of the clothes washer \\ \hline
        in.clothes\_washer\_presence & Presence of clothes washer \\ \hline
        in.clothes\_washer\_usage\_level & Clothes washer energy usage level multiplier \\ \hline
        in.cooking\_range & Presence and fuel type of the cooking range \\ \hline
        in.cooking\_range\_usage\_level & Cooking range energy usage level multiplier \\ \hline
        in.cooling\_setpoint & Base cooling setpoint with no offset applied \\ \hline
        in.cooling\_setpoint\_has\_offset & Presence of cooling setpoint offset \\ \hline
        in.cooling\_setpoint\_offset\_magnitude & The magnitude of cooling setpoint offset \\ \hline
        in.cooling\_setpoint\_offset\_period & The period during which the cooling setpoint offset is applied \\ \hline
        in.corridor & Type of corridor \\ \hline
        in.county & County GISJOIN identifier \\ \hline
        in.county\_and\_puma & The GISJOIN identifier for the County and the PUMA that the sample is located \\ \hline
        
        in.dehumidifier & Presence, water removal rate, and humidity setpoint of dehumidifier (not modeled in 2024.2 dataset) \\ \hline
        in.dishwasher & Presence and rated efficiency of dishwasher \\ \hline
        in.dishwasher\_usage\_level & Dishwasher energy usage level multiplier \\ \hline
        in.door\_area & Area of exterior doors \\ \hline
        in.doors & Exterior door material and properties \\ \hline
        in.duct\_leakage\_and\_insulation & Duct insulation and leakage to outside from the portion of ducts in unconditioned spaces \\ \hline
        in.duct\_location & Location of duct system \\ \hline
        in.eaves & Depth of roof eaves \\ \hline
        in.electric\_vehicle & Electric vehicle usage and efficiency \\ \hline
        
        in.energystar\_climate\_zone\_2023 & Climate zones for windows, doors, and skylights per ENERGY STAR guidelines as of 2023 \\ \hline
        in.federal\_poverty\_level & Federal poverty level of the household occupying the housing unit \\ \hline
        in.generation\_and\_emissions\_assessment\_region & Generation and carbon emissions assessment (GEA) region from Cambium 2022 \\ \hline
        in.geometry\_attic\_type & Type of attic \\ \hline
        in.geometry\_building\_horizontal\_location\_mf & Location of the multifamily unit horizontally within the building (left, middle, right) \\ \hline
        in.geometry\_building\_horizontal\_location\_sfa & Location of the single-family attached unit horizontally within the building (left, middle, right) \\ \hline
        in.geometry\_building\_level\_mf & Location of the multifamily unit vertically within the building (bottom, middle, top) \\ \hline
        in.geometry\_building\_number\_units\_mf & Number of units in the multifamily building in which the housing unit is located \\ \hline
        in.geometry\_building\_number\_units\_sfa & Number of units in the single-family attached building in which housing unit is located \\ \hline
        in.geometry\_building\_type\_acs & American Community Survey building type \\ \hline
        in.geometry\_building\_type\_height & RECS 2009 building type with multifamily buildings split out by low-rise, mid-rise, and high-rise \\ \hline
        in.geometry\_building\_type\_recs & PUMS 2019 building type \\ \hline
        in.geometry\_floor\_area & Finished floor area bin (American Housing Survey) \\ \hline
        in.geometry\_floor\_area\_bin & Finished floor area bin \\ \hline
        in.geometry\_foundation\_type & Type of building foundation \\ \hline
        in.geometry\_garage & Presence and size of an attached garage \\ \hline
        in.geometry\_space\_combination & Valid combinations of building type, building level MF, attic, foundation, and garage \\ \hline
        in.geometry\_stories & Number of building stories in which housing unit is located \\ \hline
        in.geometry\_stories\_low\_rise & Number of building stories for low rise building in which housing unit is located \\ \hline
        in.geometry\_story\_bin & The building in which housing unit is located has 8 or more versus fewer than 8 stories \\ \hline
        in.geometry\_wall\_exterior\_finish & Exterior wall finish material and color \\ \hline
        in.geometry\_wall\_type & Exterior wall material \\ \hline
        in.ground\_thermal\_conductivity & The thermal conductivity of the ground using in foundation and geothermal heat pump heat transfer calculations \\ \hline
        in.has\_pv & Presence of rooftop PV \\ \hline
        in.heating\_fuel & Fuel used for primary heating \\ \hline
        in.heating\_setpoint & Baseline heating setpoint with no offset applied \\ \hline
        in.heating\_setpoint\_has\_offset & Presence of heating setpoint offset \\ \hline
        in.heating\_setpoint\_offset\_magnitude & The magnitude of heating setpoint offset \\ \hline
        in.heating\_setpoint\_offset\_period & The period during which the heating setpoint offset is applied \\ \hline
        in.holiday\_lighting & Presence, energy usage, and schedule of holiday lighting \\ \hline
        in.hot\_water\_distribution & Hot water piping material and insulation level \\ \hline
        in.hot\_water\_fixtures & Hot water fixture usage and flow levels \\ \hline
        in.household\_has\_tribal\_persons & The housing unit houses at least one Tribal person \\ \hline
        in.hvac\_cooling\_efficiency & Presence and efficiency of cooling system \\ \hline
        in.hvac\_cooling\_partial\_space\_conditioning & The fraction of the finished floor area that is cooled by the cooling system \\ \hline
        in.hvac\_cooling\_type & Presence and type of cooling system \\ \hline
        in.hvac\_has\_ducts & Presence of ducts \\ \hline
        in.hvac\_has\_shared\_system & Presence of shared HVAC system \\ \hline
        in.hvac\_has\_zonal\_electric\_heating & Presence of electric baseboard heating \\ \hline
        in.hvac\_heating\_efficiency & Presence and efficiency of primary heating system \\ \hline
        in.hvac\_heating\_type & Presence and type of primary heating system \\ \hline
        in.hvac\_heating\_type\_and\_fuel & Presence, type, and fuel of primary heating system \\ \hline
        in.hvac\_secondary\_heating\_efficiency & Presence and efficiency of secondary heating system \\ \hline
        in.hvac\_secondary\_heating\_fuel & Secondary HVAC system heating type and fuel \\ \hline
        in.hvac\_secondary\_heating\_partial\_space\_conditioning & Fraction of heat load served by secondary heating system (not modeled in 2024.2 baseline dataset) \\ \hline
        in.hvac\_shared\_efficiencies & Presence and efficiency of shared HVAC system \\ \hline
        in.hvac\_system\_is\_faulted & Not used \\ \hline
        in.hvac\_system\_single\_speed\_ac\_airflow & Not used \\ \hline
        in.hvac\_system\_single\_speed\_ac\_charge & Not used \\ \hline
        in.hvac\_system\_single\_speed\_ashp\_airflow & Not used \\ \hline
        in.hvac\_system\_single\_speed\_ashp\_charge & Not used \\ \hline
        in.income & Income bin of the household occupying the housing unit \\ \hline
        in.income\_recs\_2015 & Income bin of the household occupying the housing unit aligned with the 2015 U.S. EIA RECS \\ \hline
        in.income\_recs\_2020 & Income bin of the household occupying the housing unit aligned with the 2020 U.S. EIA RECS \\ \hline
        in.infiltration & Air leakage rates for the living and garage spaces \\ \hline
        in.insulation\_ceiling & Ceiling insulation level (between the living space and unconditioned attic) \\ \hline
        in.insulation\_floor & Floor insulation level \\ \hline
        in.insulation\_foundation\_wall & Foundation walls insulation level \\ \hline
        in.insulation\_rim\_joist & Insulation level for rim joists \\ \hline
        in.insulation\_roof & Finished roof insulation level between roof and conditioned space \\ \hline
        in.insulation\_slab & Slab insulation level \\ \hline
        in.insulation\_wall & Wall construction type and insulation level \\ \hline
        in.interior\_shading & Fraction of the window area shaded from the interior in the summer and winter \\ \hline
        in.iso\_rto\_region & ISO or RTO region \\ \hline
        in.lighting & Fraction of lighting types \\ \hline
        in.lighting\_interior\_use & Interior lighting usage relative to the national average \\ \hline
        in.lighting\_other\_use & Exterior and garage lighting usage relative to the national average \\ \hline
        in.location\_region & Custom ResStock region \\ \hline
        in.mechanical\_ventilation & Mechanical ventilation type and efficiency \\ \hline
        in.misc\_extra\_refrigerator & Presence and rated efficiency of extra refrigerator \\ \hline
        in.misc\_freezer & Presence and rated efficiency of standalone freezer \\ \hline
        in.misc\_gas\_fireplace & Presence of gas fireplace \\ \hline
        in.misc\_gas\_grill & Presence of gas grill \\ \hline
        in.misc\_gas\_lighting & Presence of exterior gas lighting \\ \hline
        in.misc\_hot\_tub\_spa & Presence and fuel type of hot tub \\ \hline
        in.misc\_pool & Presence of pool \\ \hline
        in.misc\_pool\_heater & Presence and fuel type of pool heater \\ \hline
        in.misc\_pool\_pump & Presence and size of pool pump \\ \hline
        in.misc\_well\_pump & Presence and efficiency of well pump \\ \hline
        in.natural\_ventilation & Schedule of natural ventilation from windows \\ \hline
        in.neighbors & Presence and distance between the housing unit and the nearest neighbors to the left and right. \\ \hline
        in.occupants & The number of occupants living in the housing unit \\ \hline
        in.orientation & Orientation of the building \\ \hline
        in.overhangs & Presence, depth, and location of window overhangs \\ \hline
        in.plug\_load\_diversity & Plug load diversity multiplier relative to the national average \\ \hline
        in.plug\_loads & Plug load usage level relative to the national average \\ \hline
        in.puma & The  2010 U.S. Census PUMA where the sample is located \\ \hline
        in.puma\_metro\_status & The PUMA metropolitan status \\ \hline
        in.pv\_orientation & Presence and orientation of rooftop PV system \\ \hline
        in.pv\_system\_size & Presence and size of rooftop PV system \\ \hline
        in.radiant\_barrier & Presence of radiant barrier in attic \\ \hline
        in.range\_spot\_vent\_hour & Range spot ventilation daily start hour \\ \hline
        in.reeds\_balancing\_area & Regional Energy Deployment System Model balancing area \\ \hline
        in.refrigerator & The presence and rated efficiency of the primary refrigerator \\ \hline
        in.refrigerator\_usage\_level & Refrigerator energy usage level multiplier \\ \hline
        in.roof\_material & Roof material type \\ \hline
        
        in.solar\_hot\_water & Presence, size, and location of solar hot water system \\ \hline
        in.sqft & Finished floor area of the representative housing unit \\ \hline
        in.state & State \\ \hline
        in.tenure & The tenancy (owner or renter) of the household occupying the housing unit \\ \hline
        in.units\_represented & Number of housing units the building model represents (this field is no longer used) \\ \hline
        in.usage\_level & Usage of major appliances relative to the national average \\ \hline
        
        in.vacancy\_status & Presence of occupants \\ \hline
        in.vintage & Range in which the building was constructed \\ \hline
        in.vintage\_acs & Range in which the building was constructed using ACS bins \\ \hline
        in.water\_heater\_efficiency & Efficiency, type, and heating fuel of water heater \\ \hline
        in.water\_heater\_fuel & Water heater fuel \\ \hline
        in.water\_heater\_in\_unit & Individual water heater present or not present in the housing unit that solely serves the specific housing unit \\ \hline
        in.water\_heater\_location & Water heater location for the housing unit if applicable \\ \hline
        in.weather\_file\_city & City of weather file \\ \hline
        
        in.window\_areas & Window to wall ratios of the front, back, left, and right walls  \\ \hline
        in.windows & Construction type and efficiency levels of windows \\
\end{customLongTable}

In addition to the housing characteristics themselves, several other ResStock outputs receive the ``in.'' prefix. These are inputs that are not provided as housing characteristics, but that are necessary inputs to the model. Most of these values are specified in the \href{https://github.com/NREL/resstock/blob/v3.2.0-2024.2/project_national/EUSSRR2_project_500k_AMY2018.yml}{project configuration file}. 

\begin{customLongTable}{ |p{6cm}|p{10.0cm}| }
{ResStock building characteristic output field names and descriptions} {table:supplmentary_input_fields}
{\textbf{Field Name} & \textbf{Field Description}} 

        in.emissions\_electricity\_folders & Relative paths of electricity emissions factor schedule files with hourly values. Paths are relative to the resources folder. If multiple scenarios, use a comma-separated list. File names must contain GEA region names. \\ \hline
        in.emissions\_electricity\_units & Electricity emissions factors units. If multiple scenarios, use a comma-separated list. Only lb/MWh and kg/MWh are allowed. \\ \hline
        in.emissions\_electricity\_values\_or\_filepaths & Electricity emissions factors values, specified as either an annual factor or an absolute/relative path to a file with hourly factors. If multiple scenarios, use a comma-separated list. \\ \hline
        in.emissions\_fossil\_fuel\_units & Fossil fuel emissions factors units. If multiple scenarios, use a comma-separated list. Only lb/MBtu and kg/MBtu are allowed. \\ \hline
        in.emissions\_fuel\_oil\_values & Fuel oil emissions factors values, specified as an annual factor. If multiple scenarios, use a comma-separated list. \\ \hline
        in.emissions\_natural\_gas\_values & Natural gas emissions factors values, specified as an annual factor. If multiple scenarios, use a comma-separated list. \\ \hline
        in.emissions\_propane\_values & Propane emissions factors values, specified as an annual factor. If multiple scenarios, use a comma-separated list. \\ \hline
        in.emissions\_scenario\_names & Names of emissions scenarios. If multiple scenarios, use a comma-separated list. \\ \hline
        %in.emissions\_types & Types of emissions (e.g., CO2e, NOx, etc.). If multiple scenarios, use a comma-separated list \\ \hline
        in.simulation\_control\_run\_period\_begin\_day\_of\_month & The starting day of the starting month for the annual run period. \\ \hline
        in.simulation\_control\_run\_period\_begin\_month & The starting month number (1 = January, 2 = February, etc.) for the annual run period. \\ \hline
        in.simulation\_control\_run\_period\_calendar\_year & The calendar year that determines the start day of week. \\ \hline
        in.simulation\_control\_run\_period\_end\_day\_of\_month & The ending day of the ending month for the annual run period. \\ \hline
        in.simulation\_control\_run\_period\_end\_month & The end month number (1 = January, 2 = February, etc.) for the annual run period. \\ \hline
        in.simulation\_control\_timestep & Value must be a divisor of 60. \\ \hline

        in.utility\_bill\_detailed\_filepaths & n/a \\ \hline
        in.utility\_bill\_electricity\_filepaths & n/a \\ \hline
        in.utility\_bill\_electricity\_fixed\_charges & Electricity utility bill monthly fixed charges. If multiple scenarios, use a comma-separated list. \\ \hline
        in.utility\_bill\_electricity\_marginal\_rates & Electricity utility bill marginal rates. If multiple scenarios, use a comma-separated list. \\ \hline
        in.utility\_bill\_fuel\_oil\_fixed\_charges & Fuel oil utility bill monthly fixed charges. If multiple scenarios, use a comma-separated list. \\ \hline
        in.utility\_bill\_fuel\_oil\_marginal\_rates & Fuel oil utility bill marginal rates. If multiple scenarios, use a comma-separated list. \\ \hline
        in.utility\_bill\_natural\_gas\_fixed\_charges & Natural gas utility bill monthly fixed charges. If multiple scenarios, use a comma-separated list. \\ \hline
        in.utility\_bill\_natural\_gas\_marginal\_rates & Natural gas utility bill marginal rates. If multiple scenarios, use a comma-separated list. \\ \hline
        in.utility\_bill\_propane\_fixed\_charges & Propane utility bill monthly fixed charges. If multiple scenarios, use a comma-separated list. \\ \hline
        in.utility\_bill\_propane\_marginal\_rates & Propane utility bill marginal rates. If multiple scenarios, use a comma-separated list. \\ \hline
        in.utility\_bill\_scenario\_names & Names of utility bill scenarios. If multiple scenarios, use a comma-separated list. If multiple scenarios, use a comma-separated list. \\ \hline
        in.utility\_bill\_simple\_filepaths & Relative paths of simple utility rates. Paths are relative to the resources folder. If multiple scenarios, use a comma-separated list. \\ \hline

        in.weather\_file\_latitude & Latitude of location of weather file. \\ \hline
        in.weather\_file\_longitude & Longitude of location of weather file. \\ \hline
    \end{customLongTable}

\section{Energy Consumption by Fuel and End Use}
The ResStock workflow models each housing unit sample in OpenStudio and EnergyPlus to produce energy consumption of each model at subhourly time steps and then processes the results into ResStock outputs. These outputs are produced by fuel type (e.g., electricity, natural gas) and end use (e.g., lighting, heating). Totals are calculated for each fuel and for all fuels combined. For electricity and all fuels combined, net totals are also calculated, which incorporate the impacts of on-site photovoltaics. The full list of energy consumption outputs from the ResStock workflow is available in Table \ref{table:results_output_fields}. Note that to date, ResStock has included only consumption of electricity, natural gas, propane, and fuel oil---excluding on-site consumption, wood, coal, or other fuels in its modeling, although the workflow is set up to include these additional fuels. Future releases will likely include wood. 

As an output, the energy consumption results are all preceded with an ``out.'' prefix, followed by either the fuel type (e.g., ``out.electricity.'') or ``total.'' if it is an aggregate of all fuel types, and then the end use, ``net.''  or ``total.'' if it is an aggregate of all end uses in that fuel type.
%%%% Bill calculations are really confusing and need to be better covered either here or OS-HPXML (or both). OS-HPXML documentation is lacking

\begin{customLongTable}{ |p{6.cm}|p{1.8cm}|p{8.2cm}| }
{ResStock energy output field names, units, and descriptions} {table:results_output_fields} 
{\textbf{Field Name} & \textbf{Units} & \textbf{Field Description}} 
        out.electricity.ceiling\_fan.energy\_consumption.kwh & kWh & Electricity consumed by ceiling fans \\ \hline
        out.electricity.clothes\_dryer.energy\_consumption.kwh & kWh & Electricity consumed by clothes dryers \\ \hline
        out.electricity.clothes\_washer.energy\_consumption.kwh & kWh & Electricity consumed by clothes washers \\ \hline
        out.electricity.cooling.energy\_consumption.kwh & kWh & Electricity consumed by cooling systems; excludes usage by fans/pumps \\ \hline
        out.electricity.cooling\_fans\_pumps.energy\_consumption.kwh & kWh & Electricity consumed by supply fan (air distribution) or circulating pump (geothermal loop) during cooling \\ \hline
        out.electricity.dishwasher.energy\_consumption.kwh & kWh & Electricity consumed by dishwashers \\ \hline
        out.electricity.freezer.energy\_consumption.kwh & kWh & Electricity consumed by standalone freezers \\ \hline
        out.electricity.heating.energy\_consumption.kwh & kWh & Electricity consumed by heating systems; excludes usage by fans/pumps \\ \hline
        out.electricity.heating\_fans\_pumps.energy\_consumption.kwh & kWh & Electricity consumed by supply fan (air distribution) or circulating pump (hydronic distribution or geothermal loop) during heating \\ \hline
        out.electricity.heating\_hp\_bkup.energy\_consumption.kwh & kWh & Electricity consumed by heat pump backup; excludes usage by heat pump backup fans/pumps \\ \hline
        out.electricity.heating\_hp\_bkup\_fa.energy\_consumption.kwh & kWh & Electricity consumed by supply fan (air distribution) or circulating pump (hydronic distribution) during heat pump backup \\ \hline
        out.electricity.hot\_water.energy\_consumption.kwh & kWh & Electricity consumed by hot water system excludes recirculation pump and solar thermal pump \\ \hline
        out.electricity.lighting\_exterior.energy\_consumption.kwh & kWh & Electricity consumed by exterior lighting \\ \hline
        out.electricity.lighting\_garage.energy\_consumption.kwh & kWh & Electricity consumed by lighting in the garage \\ \hline
        out.electricity.lighting\_interior.energy\_consumption.kwh & kWh & Electricity consumed by interior lighting \\ \hline
        out.electricity.mech\_vent.energy\_consumption.kwh & kWh & Electricity consumed by mechanical ventilation system \\ \hline
        out.electricity.net.energy\_consumption.kwh & kWh & Total electricity consumed subtracts any power produced by PV or generators \\ \hline
        out.electricity.permanent\_spa\_heat.energy\_consumption.kwh & kWh & Electricity consumed by spa heating \\ \hline
        out.electricity.permanent\_spa\_pump.energy\_consumption.kwh & kWh & Electricity consumed by spa pump \\ \hline
        out.electricity.plug\_loads.energy\_consumption.kwh & kWh & Electricity consumed by plug loads not elsewhere accounted for \\ \hline
        out.electricity.pool\_heater.energy\_consumption.kwh & kWh & Electricity consumed by pool heaters \\ \hline
        out.electricity.pool\_pump.energy\_consumption.kwh & kWh & Electricity consumed by pool pumps \\ \hline
        out.electricity.pv.energy\_consumption.kwh & kWh & Energy produced by rooftop PV systems. Negative value for any power produced. \\ \hline
        out.electricity.range\_oven.energy\_consumption.kwh & kWh & Electricity consumed by range and oven \\ \hline
        out.electricity.refrigerator.energy\_consumption.kwh & kWh & Electricity consumed by refrigerators \\ \hline
        out.electricity.total.energy\_consumption.kwh & kWh & Total electricity consumed \\ \hline
        out.natural\_gas.clothes\_dryer.energy\_consumption.kwh & kWh & Natural gas consumed by natural gas clothes dryers \\ \hline
        out.natural\_gas.fireplace.energy\_consumption.kwh & kWh & Natural gas consumed by natural gas fireplaces \\ \hline
        out.natural\_gas.grill.energy\_consumption.kwh & kWh & Natural gas consumed by natural gas grills \\ \hline
        out.natural\_gas.heating.energy\_consumption.kwh & kWh & Natural gas consumed by natural gas heating systems \\ \hline
        out.natural\_gas.heating\_hp\_bkup.energy\_consumption.kwh & kWh & Natural gas consumed by heat pump backup \\ \hline
        out.natural\_gas.hot\_water.energy\_consumption.kwh & kWh & Natural gas consumed by natural gas hot water systems \\ \hline
        out.natural\_gas.lighting.energy\_consumption.kwh & kWh & Natural gas consumed by natural gas lighting \\ \hline
        out.natural\_gas.permanent\_spa\_heat.energy\_consumption.kwh & kWh & Natural gas consumed by spa heating \\ \hline
        out.natural\_gas.permanent\_spa\_pump.energy\_consumption.kwh & kWh & Natural gas consumed by spa pump \\ \hline
        out.natural\_gas.pool\_heater.energy\_consumption.kwh & kWh & Natural gas consumed by natural gas pool heaters \\ \hline
        out.natural\_gas.range\_oven.energy\_consumption.kwh & kWh & Natural gas consumed by natural gas range and oven \\ \hline
        out.natural\_gas.total.energy\_consumption.kwh & kWh & Total natural gas consumed \\ \hline
        out.params.size\_cooling\_system\_primary\_k\_btu\_h & kBtu/h & Size of primary cooling system \\ \hline
        out.params.size\_heat\_pump\_backup\_primary\_k\_btu\_h & kBtu/h & Size of primary heat pump backup \\ \hline
        out.params.size\_heating\_system\_primary\_k\_btu\_h & kBtu/h & Size of primary heating system \\ \hline
        out.params.size\_heating\_system\_secondary\_k\_btu\_h & kBtu/h & Size of secondary heating system \\ \hline
        out.propane.clothes\_dryer.energy\_consumption.kwh & kWh & Propane consumed by propane clothes dryers \\ \hline
        out.propane.heating.energy\_consumption.kwh & kWh & Propane consumed by propane heating systems \\ \hline
        out.propane.heating\_hp\_bkup.energy\_consumption.kwh & kWh & Propane consumed by heat pump backup \\ \hline
        out.propane.hot\_water.energy\_consumption.kwh & kWh & Propane consumed by propane hot water systems \\ \hline
        out.propane.range\_oven.energy\_consumption.kwh & kWh & Propane consumed by propane range and oven \\ \hline
        out.propane.total.energy\_consumption.kwh & kWh & Total propane energy consumed \\ \hline
        out.site\_energy.net.energy\_consumption.kwh & kWh & Total site energy consumed subtracts any power produced by PV or generators \\ \hline
        out.site\_energy.total.energy\_consumption.kwh & kWh & Total site energy consumed \\ 
\end{customLongTable}

\section{Cost Multipliers}
The ResStock workflow calculates and outputs certain values to support the calculation of costs of implementation of upgrades and upgrade packages. These are values that are available to scale measure implementation cost---for example, the total square feet of exterior window area in a housing unit model sample that can be multiplied by a user-provided window cost per square foot for a specific measure to get a per-housing-unit measure cost. They are calculated from the building energy models. ResStock currently provides 21 such values, as listed in Table \ref{table:cost_multipliers}. In published ResStock datasets, these cost multipliers are the fields that use the ``out.params'' prefix, for example ``out.params.window\_area\_ft\_2.'' The exception is that ``upgrade\_costs.floor\_area\_conditioned\_ft\_2'' has to-date been published as ``in.sqft.''

\begin{longtable}[]{ |p{6.cm}|p{2.5cm}|p{7.2cm}| }
\caption{ResStock cost multiplier output field names, units, and descriptions} \label{table:cost_multipliers} \\
\toprule\noalign{}
\textbf{Field Name} & \textbf{Units} & \textbf{Description} \\
\midrule\noalign{}
\endhead
\bottomrule\noalign{}
\endlastfoot
upgrade\_costs.door\_area\_ft\_2 & ft\textsuperscript{2} & Door Area \\ \hline
upgrade\_costs.duct\_unconditioned\_surface\_area\_ft\_2 & ft\textsuperscript{2}& Duct Unconditioned Surface Area \\ \hline
upgrade\_costs.floor\_area\_attic\_ft\_2 & ft\textsuperscript{2}& Floor Area, Attic \\ \hline
upgrade\_costs.floor\_area\_attic\_insulation\_increase\_ft\_2\_delta\_r\_value & ft\textsuperscript{2} * $\Delta$ R-value& Floor Area, Attic * Insulation Increase \\ \hline
upgrade\_costs.floor\_area\_conditioned\_ft\_2 & ft\textsuperscript{2}& Floor Area, Conditioned \\ \hline
upgrade\_costs.floor\_area\_conditioned\_infiltration\_reduction\_ft\_2\_delta\_ach\_50 & ft\textsuperscript{2} * $\Delta$ ACH50& Floor Area, Conditioned * Infiltration Reduction \\ \hline
upgrade\_costs.floor\_area\_foundation\_ft\_2 & ft\textsuperscript{2}& Floor Area, Foundation \\ \hline
upgrade\_costs.floor\_area\_lighting\_ft\_2 & ft\textsuperscript{2}& Floor Area, Lighting \\ \hline
upgrade\_costs.flow\_rate\_mechanical\_ventilation\_cfm  & cfm & Flow Rate, Mechanical Ventilation \\ \hline
upgrade\_costs.rim\_joist\_area\_above\_grade\_exterior\_ft\_2  & ft\textsuperscript{2}& Rim Joist Area, Above-Grade, Exterior \\ \hline
upgrade\_costs.roof\_area\_ft\_2  &  ft\textsuperscript{2}& Roof Area \\ \hline
upgrade\_costs.size\_cooling\_system\_primary\_k\_btu\_h  & kBtu/h & Size, Cooling System Primary \\ \hline
upgrade\_costs.size\_heat\_pump\_backup\_primary\_k\_btu\_h & kBtu/h & Size, Heat Pump Backup Primary \\ \hline
upgrade\_costs.size\_heating\_system\_primary\_k\_btu\_h & kBtu/h & Size, Heating System Primary \\ \hline
upgrade\_costs.size\_heating\_system\_secondary\_k\_btu\_h & kBtu/h & Size, Heating System Secondary \\ \hline
upgrade\_costs.size\_water\_heater\_gal & gal & Size, Water Heater \\ \hline
upgrade\_costs.slab\_perimeter\_exposed\_conditioned\_ft & ft & Slab Perimeter, Exposed, Conditioned \\ \hline
upgrade\_costs.wall\_area\_above\_grade\_conditioned\_ft\_2 & ft\textsuperscript{2}& Wall Area, Above-Grade, Conditioned \\ \hline
upgrade\_costs.wall\_area\_above\_grade\_exterior\_ft\_2 & ft\textsuperscript{2}& Wall Area, Above-Grade, Exterior \\ \hline
upgrade\_costs.wall\_area\_below\_grade\_ft\_2 & ft\textsuperscript{2}& Wall Area, Below-Grade \\ \hline
upgrade\_costs.window\_area\_ft\_2 & ft\textsuperscript{2}& Window Area
\end{longtable}

\section{Emissions}
%Standard convention of how we do this. Methodology. 
The ResStock workflow includes the capability of including emissions factors as supplemental inputs, which are then used to calculate the corresponding emissions outputs. These outputs are generated by end use and fuel type, as well as per-fuel totals. Total emissions impacts of a measure across fuels are calculated from ResStock results.

\subsection{Emissions From On-Site Combustion (Scope 1)}
The ResStock workflow currently accepts emissions factors for non-electric energy consumption as annual values only. Our typical approach is to use the values from Table 7.1.2(1) of PDS-01 of BSR/RESNET/ICCC 301 Addendum B, CO\textsubscript{2} Index (\cite{RESNET2022}), which account for both combustion and pre-combustion (e.g., methane leakage) impacts. These are 147.3 lb/MMBtu (228.5 kg/MWh) for natural gas, 177.8 lb/MMBtu (275.8 kg/MWh) for propane, and 195.9 lb/MMBtu (303.9 kg/MWh) for fuel oil. ResStock then outputs the associated carbon-equivalent emissions for every fuel and end-use combination. 

If multiple emissions scenarios are being run, ResStock will output values for each fuel for each end use for each scenario, even if that particular fuel's emissions factors do not differ between scenarios. This occurs frequently when we run multiple electricity emissions scenarios with different values for emissions factors related to electricity generation that all use the same emissions factors for on-site non-electric fuel consumption. The full list of emissions output fields from non-electric energy consumption is in Table \ref{table:emissions_output}. In published ResStock datasets, we convert the units on these results to kilograms of CO\textsubscript{2} equivalent emissions. These are the fields in public datasets that begin ``out.emissions.natural\_gas,'' ``out.emissions.propane,'' and ``out.emissions.fuel\_oil.''

\subsection{Emissions From Electricity Generation (Scope 2)}
The ResStock workflow currently accepts emissions factors for electricity consumption as either annual or hourly values. We have used several approaches in selecting electricity emissions factors for use. Our most commonly used approach is to include multiple scenarios, generally relying on data from NREL's Cambium database (\cite{Cambium2023}). When using Cambium data, we use multiple standard scenarios (potential futures of the electric grid) as a type of sensitivity. ResStock releases generally have three long-run marginal emissions scenarios with a computed a levelized factor over a 15 or 25 year lifetime with a 3\% discount rate, starting a few years in the future. We then use timeseries (hourly) data at the GEA geographic resolution. 

The full list of emissions output fields from electric energy consumption is in Table \ref{table:emissions_output}. In published ResStock datasets, we convert the units on these results to kilograms of CO\textsubscript{2} equivalent emissions. These are the fields in public datasets that begin ``out.emissions.electricity.''

\begin{customLongTable}{ |p{6.cm}|p{1.8cm}|p{8.2cm}| }
{ResStock utility bill output field names, units, and descriptions} {table:emissions_output} 
{\textbf{Field Name} & \textbf{Units} & \textbf{Field Description}} 
out.emissions.all\_fuels.lrmer\_high\_re\_cost\_15.co2e\_kg & co2e\_kg & Emissions using on-site fossil fuel rates from ANSI/RESNET and electricity using Cambium 2022 high renewable cost scenario with long run marginal emissions rate levelized by 3\% over 15 years \\ \hline
        out.emissions.all\_fuels.lrmer\_low\_re\_cost\_15.co2e\_kg & co2e\_kg & Emissions using on-site fossil fuel rates from ANSI/RESNET and electricity using Cambium 2022 low renewable cost scenario with long run marginal emissions rate levelized by 3\% over 15 years \\ \hline
        out.emissions.all\_fuels.lrmer\_mid\_case\_15.co2e\_kg & co2e\_kg & Emissions using on-site fossil fuel rates from ANSI/RESNET and electricity using Cambium 2022 mid-case scenario with long run marginal emissions rate levelized by 3\% over 15 years \\ \hline
        out.emissions.all\_fuels.lrmer\_mid\_case\_25.co2e\_kg & co2e\_kg & Emissions using on-site fossil fuel rates from ANSI/RESNET and electricity using Cambium 2022 mid-case scenario with long run marginal emissions rate levelized by 3\% over 25 years \\ \hline
        out.emissions.electricity.lrmer\_high\_re\_cost\_15.co2e\_kg & co2e\_kg & Emissions from electricity under Cambium 2022 high renewable cost scenario with long run marginal emissions rate levelized by 3\% over 15 years \\ \hline
        out.emissions.electricity.lrmer\_low\_re\_cost\_15.co2e\_kg & co2e\_kg & Emissions from electricity under Cambium 2022 low renewable cost scenario with long run marginal emissions rate levelized by 3\% over 15 years \\ \hline
        out.emissions.electricity.lrmer\_mid\_case\_15.co2e\_kg & co2e\_kg & Emissions from electricity under Cambium 2022 mid-case scenario with long run marginal emissions rate levelized by 3\% over 15 years \\ \hline
        out.emissions.electricity.lrmer\_mid\_case\_25.co2e\_kg & co2e\_kg & Emissions from electricity under Cambium 2022 mid-case scenario with long run marginal emissions rate levelized by 3\% over 25 years \\ \hline
        out.emissions.fuel\_oil.lrmer\_high\_re\_cost\_15.co2e\_kg & co2e\_kg & Emissions from on-site fuel oil consumption using ANSI/RESNET data \\ \hline
        out.emissions.fuel\_oil.lrmer\_low\_re\_cost\_15.co2e\_kg & co2e\_kg & Emissions from on-site fuel oil consumption using ANSI/RESNET data \\ \hline
        out.emissions.fuel\_oil.lrmer\_mid\_case\_15.co2e\_kg & co2e\_kg & Emissions from on-site fuel oil consumption using ANSI/RESNET data \\ \hline
        out.emissions.fuel\_oil.lrmer\_mid\_case\_25.co2e\_kg & co2e\_kg & Emissions from on-site fuel oil consumption using ANSI/RESNET data \\ \hline
        out.emissions.natural\_gas.lrmer\_high\_re\_cost\_15.co2e\_kg & co2e\_kg & Emissions from on-site natural gas consumption using ANSI/RESNET data \\ \hline
        out.emissions.natural\_gas.lrmer\_low\_re\_cost\_15.co2e\_kg & co2e\_kg & Emissions from on-site natural gas consumption using ANSI/RESNET data \\ \hline
        out.emissions.natural\_gas.lrmer\_mid\_case\_15.co2e\_kg & co2e\_kg & Emissions from on-site natural gas consumption using ANSI/RESNET data \\ \hline
        out.emissions.natural\_gas.lrmer\_mid\_case\_25.co2e\_kg & co2e\_kg & Emissions from on-site natural gas consumption using ANSI/RESNET data \\ \hline
        out.emissions.propane.lrmer\_high\_re\_cost\_15.co2e\_kg & co2e\_kg & Emissions from on-site propane consumption using ANSI/RESNET data \\ \hline
        out.emissions.propane.lrmer\_low\_re\_cost\_15.co2e\_kg & co2e\_kg & Emissions from on-site propane consumption using ANSI/RESNET data \\ \hline
        out.emissions.propane.lrmer\_mid\_case\_15.co2e\_kg & co2e\_kg & Emissions from on-site propane consumption using ANSI/RESNET data \\ \hline
        out.emissions.propane.lrmer\_mid\_case\_25.co2e\_kg & co2e\_kg & Emissions from on-site propane consumption using ANSI/RESNET data \\
\end{customLongTable}

\section{Utility Bills}
The ResStock workflow is able to accept a limited range of utility rate inputs and output the corresponding utility bills. The inputs available are fixed charges and volumetric rates for each modeled fuel (typically electricity, natural gas, propane, and fuel oil). These can vary by any one housing characteristic, such as State. Multiple scenarios may be specified. For the list of utility bill outputs, see Table \ref{table:cost_multipliers}.

Our most common approach within the ResStock workflow for utility rates for electricity and natural gas is to use a single scenario, using both a flat charge and a per-consumption charge. We typically use data downloaded from \href{https://openei.org/wiki/Utility_Rate_Database}{NREL's Utility Rate Database} to calculate the customer-weighted average fixed monthly electricity charge across all utilities in the database, which comes out to approximately \$10/customer/month or \$120/customer/year. For natural gas, we use the American Gas Association's 2015 value of \$11.25/customer/month for the fixed portion of the utility bill (generally referred to as the ``customer charge''). We additionally use EIA state-level information on revenue, sales, and number of customers to calculate the average per-consumption electricity and natural gas rate for each state with the fixed charges removed. The total electricity and natural gas bills for each housing unit model sample are then calculated by summing the flat charge and the product of the per-consumption charge and the total consumption. For example, this process may result in a flat charge of \$120/year and a per-consumption electricity rate of \$0.20/kWh---for a housing unit with a full year electricity consumption of 3000 kWh, the total electricity bill would be \$120 + (3,000 kWh * \$0.20/kWh) = \$720. If a housing unit has uses no natural gas for the year, the flat charge is not included as it is presumed not to have natural gas service. 

Our most common approach within the ResStock workflow for utility rates for propane and fuel oil is to use a single scenario using weekly volumetric rate data from the EIA, averaged over a year (either a calendar year or a winter). When state-level data are not available, we use data from the state's Petroleum Administration for Defense Districts (PADD) region. When PADD region data are not available, we use U.S. national average values. We then calculate propane and fuel oil bills by multiplying the fuel consumption for each housing unit model by the volumetric rate for that fuel and state. An example of this is in ResStock dataset 2024.1 \citep{Present2024}.

In public ResStock datasets, these fields are prefixed with ``out.bills.''

We do not currently commonly model other on-site fuel use in ResStock, such as on-site use of wood or coal, or district steam, and we do not have a common approach for calculating associated bills.%What about district steam?

\begin{longtable}[]{ |p{8.5cm}|p{1cm}|p{6.5cm}| }
\caption{ResStock utility bill output field names, units, and descriptions} \label{table:hc_cost_multipliers} \\
\toprule\noalign{}
\textbf{Field Name} & \textbf{Units} & \textbf{Description} \\
\midrule\noalign{}
\endhead
\bottomrule\noalign{}
\endlastfoot
        out.bills.all\_fuels.usd & \$ & Annual total charges for electricity, fuel oil, natural gas, and propane \\ \hline
        out.bills.electricity.usd & \$ & Annual total charges for electricity \\ \hline
        out.bills.fuel\_oil.usd & \$ & Annual total charges for fuel oil \\ \hline
        out.bills.natural\_gas.usd & \$ & Annual total charges for natural gas \\ \hline
        out.bills.propane.usd & \$ & Annual total charges for propane
\end{longtable}

ResStock results can also be used as inputs to calculate utility bills using other rate structures, such as time-of-use electricity rates or tiered rates. However, these calculations are currently not included in the primary ResStock workflow or any published datasets.
% \section{PRIORITY C NEC Load Calculations}
\section{Energy Burden}
Energy burden is calculated as the ratio of two values from ResStock data. The first is the energy bill, which is calculated for each individual housing unit model in an analysis as described in the above section. The second is the representative household income, which is converted from income bin and other household characteristics described in Section \ref{resstock_input_energy_burden}. The energy burden is the energy bill total divided by the representative household income, typically expressed as a percentage. As the utility bill does not account for financial assistance such as LIHEAP, energy burden may be overestimated for low-income households who may qualify for those assistance. Additionally, the representative income is in 2019 USD, reflecting the vintage of PUMS used to derived the data. The dollar value of the utility bill is typically more recent and thus may not align with that of the income denominator.

When this field is available in public datasets, it begins with ``out.energy\_burden.''
% \section{PRIORITY B Costs?}
\section{Other Outputs}
% \ stuff like loads, unmet hours, hot water, etc.
The ResStock workflow can optionally output any variable available in EnergyPlus. We regularly use this capability for a variety of purposes, including modeling improvement and validation, debugging, special purpose variables for specific projects, and additional variables to publish as part of public datasets. 

Table \ref{tab:other_outputs} shows a list of output variables that are not covered in other headings in this section but that we have typically included in many of our recent public datasets, where they begin with ``out.x.''

\begin{customLongTable}{ |p{6.cm}|p{1.8cm}|p{8.2cm}| }
{Other ResStock results output field names, units, and descriptions} {tab:other_outputs} 
{\textbf{Field Name} & \textbf{Units} & \textbf{Field Description}} 
        out.electricity.summer.peak.kw & kW & Maximum power value in Jun/Jul/Aug \\ \hline
        out.electricity.winter.peak.kw & kW & Maximum power value in Dec/Jan/Feb \\ \hline
        out.hot\_water.clothes\_washer.gal & gal & Hot water consumed by clothes washer \\ \hline
        out.hot\_water.dishwasher.gal & gal & Hot water consumed by dishwasher \\ \hline
        out.hot\_water.distribution\_waste.gal & gal & Hot water consumed by water distribution system (water remaining in the pipe) \\ \hline
        out.hot\_water.fixtures.gal & gal & Hot water consumed by showers, sinks, and baths \\ \hline
        out.params.size\_water\_heater\_gal & gal & Size of water heater \\ \hline
        out.load.cooling.energy\_delivered.kbtu & kbtu & Total energy delivered by cooling system includes HVAC distribution losses \\ \hline
        out.load.cooling.peak.kbtu\_hr & kbtu/hr & Maximum cooling load delivered by cooling system includes HVAC distribution losses \\ \hline
        out.load.heating.energy\_delivered.kbtu & kbtu & Total energy delivered by heating system includes HVAC distribution losses \\ \hline
        out.load.heating.peak.kbtu\_hr & kbtu/hr & Maximum heating energy delivered by heating system includes HVAC distribution losses over a 60 min time period \\ \hline
        out.load.hot\_water.energy\_delivered.kbtu & kbtu & Total energy delivered by hot water system includes contributions by desuperheaters or solar thermal systems \\ \hline
        out.unmet\_hours.cooling.hour & hr & Number of hours where the cooling setpoint is not maintained \\ \hline
        out.unmet\_hours.heating.hour & hr & Number of hours where the heating setpoint is not maintained \\ \hline
        weight & n/a & the number of housing units the sample represents \\
\end{customLongTable}
