%%%%%%SECTION NO LONGER USED
\chapter{Running ResStock}
\section{BuildStock Batch}
%What is needed to specify an upgrade? Apply logic conventions. Use of elements &,*.
% Reference the repository. Mention different ways to run ResStock through local, NREL HPC, and cloud resources. Introduce concept of the project file.
\href{https://github.com/NREL/buildstockbatch/tree/v2023.10.0}{BuildStock Batch} is a software designed to run and manage large-scale simulations for ResStock and ComStock on different computing platforms, which include personal computers, NREL’s high performance computers (HPC), and cloud computing via Amazon Web Services (AWS) or Google Cloud Platform (GCP). BuildStock Batch lets the users easily configure the simulations using a project configuration file. The project configuration file defines basic run parameters such as what type of simulations to run (e.g., baseline or upgrade scenarios), simulation timestep (e.g. hourly or 15-min resolution), weather year, sample count, and the simulation output variables to record.  When using the NREL HPC or cloud workflows, the configuration file also specifies the amount of computing resources and time required for running the simulations.

BuildStock Batch provides the workflow to setup the folder structure, facilitate the execution of the sampling and model simulations. To run the simulations efficiently, BuildStock Batch manages the computing resources at every stage of the workflow. The software batches the simulations by breaking them up into smaller jobs to be run by parallel processors or multiple computers in a distributed manner. This is critical for enabling hundreds of thousands of simulations that are otherwise too time- and memory-intensive to process. BuildStock Batch, along with high-performance computing, is thus critical to ResStock and its standard dataset releases by making large-scale simulations possible. After the simulation completes for each model, the tool compiles the simulation output into annual summary files and timeseries files that are partitioned and separated by upgrade measures. BuildStock Batch can optionally upload the result files to AWS S3 storage, where they can be crawled into AWS Athena as database tables for query.  For more details about the ResStock workflow, see Section \ref{sec:workflow}.

\section{Project configuration file} \label{sec:project_config_file}
%What information is in the project file. What schema is used in BuildStock-Batch. Point to schema documentation. Point to national_baseline and national_upgrade yaml files. Review sections of the schema. Document where we get n_buildings_represented.
The project configuration file is the main file used to configure a ResStock run. It includes details such as sample size (i.e., number of housing units to be simulated), weather year (and corresponding EPW weather files to be used), model outputs to report, and the logic of any measures or upgrades to include. This file is the primary input for BuildStock Batch. This is informally known by the ResStock team as the "YAML" file because the file uses the Yet Another Markup Language (YAML) format. It can be used to specify runs on NREL's HPC system, cloud computing services, or on a local desktop, but all data releases are currently run using NREL's HPC.  Detailed documentation on the project configuration file is available on BuildStock Batch's  \href{https://buildstockbatch.readthedocs.io/en/v2023.10.0/project_defn.html}{documentation website}. Many examples of project configuration files are publicly available in the ResStock repository, including the project configuration files from all of our public data releases. The most recent example is the project configuration file used for the 2024.2 data release, which is \href{https://github.com/NREL/resstock/blob/v3.2.0-2024.2/project_national/EUSSRR2_project_500k_AMY2018.yml}{here}. 

The project configuration file includes all of the information necessary to set up a ResStock run. The most recent release of ResStock data, \href{https://resstock.nrel.gov/datasets}{2024.2} used 
\href{https://github.com/NREL/OpenStudio/releases/tag/v3.7.0}{OpenStudio version 3.7.0}, \href{https://github.com/NREL/buildstockbatch/tree/v2023.10.0}{BuildStock-Batch v2023.10.0 (workflow generator v2024.07.20)}, \href{https://buildstockbatch.readthedocs.io/en/v2023.10.0/index.html}{2023.10.0}, and the \href{https://github.com/NREL/resstock/tree/v3.2.0-2024.2}{v3.2.0-2024.2 tagged release} of ResStock. 
    %\item \texttt{buildstock\_directory} location of cloned ResStock repository
   % \item \texttt{project\_directory} location of the ResStock input files 
    %\item \texttt{output\_directory} location to store run results
    %\item \texttt{weather\_files\_url} or \texttt{weather\_files\_path} online or local location of the weather files for the run - must be unique for each run
The \href{https://buildstockbatch.readthedocs.io/en/v2023.10.0/project_defn.html#sampler}{sampler} portion of the project configuration file provides options for how the ResStock housing characteristics could be sampled. ResStock currently uses the \textit{Residential Quota} sampler to generate the sample for all data releases. This is a quota-based sampling method that determines the dwelling units to simulate (see Section \ref{sec:sampling_methodology} for sampling method description). It requires one argument, \textit{n\_datapoints}, the number of datapoints to sample, which is typically set to 550,000. Other samplers are \textit{Residential Quota Downselect Sampler} and \textit{Precomputed Sampler}. The downselect sampler uses quota-based sampling but filters the samples to the characteristics specified in the \textit{logic} argument block. By default, it generates \textit{n\_datapoints} samples and then apply the filtering, which results in fewer samples. The argument \textit{resample} denotes whether the sampling should repeat so that there could be \textit{n\_datapoints} filtered samples instead. The precomputed sampler is used when the user wants to pass in a pre-genereated buildstock.csv sample file via the \textit{sample\_file} key instead.
       % \item \textit{Residential Quota Downselect} sampler which adds a downselect capability to the Residential Quota sampler. This can be used to generate a sample of dwelling units to simulate that includes only a subset of all dwelling units. This can be used to, for example, run ResStock only for single-family detached dwelling units, or only for the Hot-Humid climate zone. This is not used for national data releases. %It requires three arguments.
        %\begin{itemize}
         %   \item \textit{n\_datapoints} the number of datapoints to sample
          %  \item \textit{logic} the criteria to apply to remove buildings from the simulation
           % \item \textit{resample} a boolean, which specifies whether to run the sampling twice with a goal to have the downlselected number of datapoints be equal to n\_datapoints. Downselecting with resampling samples twice, once to determine how much smaller the set of sampled buildings becomes when it is filtered down and again with a larger sample so the final set of sampled buildings is at or near the number specified in n\_datapoints. Downselecting without resampling skips that step. In this case the total sampled buildings returned will be the number left over after sampling the entire stock and then filtering down to the buildings that meet the criteria.
        % \end{itemize}
      %  \item \textit{Precomputed }sampler which skips the sampling step and inputs a user-provided set of dwelling units for simulation. It takes one argument, \textit{sample\_file}, which specifies the location of the file of dwelling units for simulation. This can be used to, for example, run ResStock multiple times with the same set of dwelling unit sample models. The \textit{buildstock.csv} file that is generated by the residential quota or residential quota downselect samplers can be used as the sample file. Although quota-based sampling is deterministic, and therefore should generate the exact same samples as long as inputs haven't changed, sometimes this option is used to ensure identical samples between runs.
   % \end{itemize}
   % \subsubsection{HPC or AWS Run Parameters}
    %\item HPC Configuration information for running the ResStock workflow on Kestrel, NREL's supercomputer. This includes:
    %\begin{itemize}
     %   \item  \textit{account} the HPC project account to use for the simulation
      %  \item \textit{n\_jobs}: the number of jobs to parallelize the simulation into which is a handful for small test runs and over 500 for full-scale runs
     %   \item \textit{minutes\_per\_sim}: the maximum allocated simulation time in minutes; we currently use 10
      %  \item \textit{sampling }configuration, namely \textit{time}, the maximum time in minutes to allocate to the sampling job; we currently use 60
       % \item \textit{postprocessing} configuration which includes:
        %\begin{itemize}
         %   \item time: maximum time in minutes to allocate to postprocessing; we usually use 800
          %  \item \textit{n\_workers}: number of nodes to parallelize the postprocessing job into. Max supported is 32. Default is 2. We currently use 30.
           % \item \textit{n\_procs}: Number of CPUs to use within each node. Max is 104. Default is 52. OOM errors can sometimes be resolved by reducing this. We've recently used 30.
           % \item node\_memory\_mb: the memory (in MB) to request per node for postprocessing. The default is 250000 which is a standard node. We currently use the default.
           % \item parquet\_memory\_mb: the size (in MB) of the combined parquet file in memory. Default is 1000. We currently use the default.
       % \end{itemize}
   % \end{itemize}
   % \item AWS configuration information for running the ResStock workflow on AWS. The ResStock team has not recently done this at scale. Please refer to the documentation for the latest information.
   % \item Postprocessing configuration options
   % \begin{itemize}
    %\item \textit{keep\_individual\_timeseries}: (optional, bool) For some use cases it is useful to keep the timeseries output for each simulation as a separate parquet file. Setting this option to \verb|true| allows that. Default is false, and all public data releases do not separate individual timeseries
   % \item  \texttt{partition\_columns}: (optional, list) Allows partitioning the output data based on some columns. The columns must match the parameters found in options\_lookup.tsv. This allows for efficient athena queries. Only recommended for moderate or large sized runs (ndatapoints > 10K). We currently typically partition large runs on \textit{State} and \textit{County}.
%\end{itemize}
The \href{https://buildstockbatch.readthedocs.io/en/v2023.10.0/project_defn.html#workflow-generator}{workflow generator} includes arguments for how the ResStock housing characteristics are sequentially transformed into OpenStudio input files and EnergyPlus models. Key inputs include the \texttt{simulation\_control\_timestep}, the timestep of the energy simulation, which is set to 15 minutes for standard data releases. Similarly, in the \texttt{simulation\_output} section of the YAML, the \texttt{timeseries\_frequency} is set to \textit{timestep}, which mean data will be reported at the 15-minute interval. 
%\begin{itemize}
 %   \item \textit{type}: we currently use the \textit{residential\_hpxml} generator for all ResStock runs
  %  \item args: there are many options for arguments, including the following sections:
   % \begin{itemize}
    %    \item build\_existing\_model: this section allows you to set 
     %   \begin{itemize}
      %      \item the \texttt{simulation\_control\_timestep} set to 15 minutes for standard data releases
       %     \item the run time period (we generally do a full year)
        %    \item the calendar year for non-AMY weather files which impacts start day of the year and the dates of daylight savings schedule adjustments. We typically use 2007. For runs using AMY weather files, the year from the EPW file will be used instead.
       % \end{itemize}
       % \item emissions: see \ref{specifying emissions and utility bill calculations}
       % \item utility\_bills: see \ref{specifying emissions and utility bill calculations}
       % \item measures
       % \item simulation\_output\_report: this section allows you to set
       % \begin{itemize}
           % \item output timeseries frequency; we usually use either \textit{timestep}, which matches the simulation timestep of 15 minutes, \textit{monthly}, or \textit{none}, which provides full-year results only
           % \item which outputs to include at the timeseries frequency, such as emissions, unmet hours, component loads, temperatures, etc.
           % \item specific EnergPlus output variables to include, such as outdoor air drybulb temperature, indoor temperatures, humidity, etc.
       % \end{itemize}
       % \item reporting\_measures
      %  \item server\_directory\_cleanup: Optionally preserve or delete various simulation output files. These arguments are passed directly to the \href{https://github.com/NREL/resstock/blob/develop/measures/ServerDirectoryCleanup/measure.xml}{ServerDirectoryCleanup} measure in resstock. Please refer to the measure arguments there to determine what to set them to in your config file. Note that the default behavior is to retain some files and remove others. See \href{https://buildstockbatch.readthedocs.io/en/v2023.10.0/workflow_generators/residential_hpxml.html#hpxml-server-dir-cleanup-defaults}{Server Directory Cleanup Defaults}.
      %  \item debug: enabling debug mode will preserver all simulation input and output files, including but not limited to: in.osm, all EnergyPlus output files, and intermediate existing and upgraded files; we keep debug off for full-scale runs
   % \end{itemize}
%\end{itemize}


% 15-min outputs, weather year(s) - build_existing_model

The project configuration file is also where certain outputs are specified, for example, greenhouse gas emission factors, measure upgrade capital costs, or utility bills. For details on selected outputs please see Section \ref{sec:resstock_outputs}.

%\subsection{Specifying emissions and utility bill calculations}
%\subsubsection{Emissions}
%Emissions factors for electricity can optionally be specified within the workflow. Currently, there is one option for this: the elec\_folder field takes the path to a folder of schedule files with hourly electricity values in kg/MWh. Folder names must contain GEA region names. ResStock currently uses the Cambium 2021-2022 GEA region definitions.  To use annual values rather than timeseries, the same value can be input for every hour. 
%Emissions factors for onsite consumption of natural gas, propane, and fuel oil can be specified directly in the project file. If they are not specified, default values are used. 
%For details on the emissions factors we typically use when running ResStock, please see \ref{emissions}.
%\subsubsection{Utility Bills}
%Utility rates for calculating utility bills can optionally be specified within the workflow. There are three options for specifying the utility rates.
%\begin{itemize}
 %   \item Input utility rates to be used for all dwelling units directly into the project specification file, using any or all of the following variables:
  %  \begin{itemize}
   %     \item elec\_fixed\_charge: monthly fixed charge for electricity
    %    \item elec\_marginal\_rate: marginal rate for electricity. Units are \$/kWh
     %   \item gas\_fixed\_charge: monthly fixed charge for natural gas
      %  \item gas\_marginal\_rate: Marginal rate for natural gas. Units are \$/therm.
        %\item propane\_fixed\_charge: monthly fixed charge for propane
      %  \item propane\_marginal\_rate: Marginal rate for propane. Units are \$/gallon.
      %  \item oil\_fixed\_charge: monthly fixed charge for propane
      %  \item oil\_marginal\_rate: marginal rate for %fuel oil. Units are \$/gallon.
     %   \item wood\_fixed\_charge: monthly fixed %charge for wood
     %   \item wood\_marginal\_rate: marginal rate for wood. Units are \$/gallon
     %   \item pv\_compensation\_type: photovoltaic compensation types. Can be NetMetering or FeedInTariff
      %  \item %pv\_net\_metering\_annual\_excess\_sellback\_rate\_type: photovoltaic net metering annual excess sellback rate. Applies if compensation type is NetMetering.
     %   \item pv\_net\_metering\_annual\_excess\_sellback\_rate: photovoltaic net metering annual excedss sellback rate. Applies if compensation type is NetMetering.
     %   \item pv\_feed\_in\_tariff\_rate: photovoltaic annual full/gross feed-in tariff rate. Applies if compensation type is FeedInTariff.
     %   \item pv\_monthly\_grid\_connection\_fee\_units: photovoltaic monthly grid connection fee units. Can be \$ or or \$/kW
     %   \item pv\_monthly\_grid\_connection\_fee\_units: photovoltaic monthly grid connection fee
   % \end{itemize}
  %  \item Provide a tsv file with all fixed charge / marginal rate/ PV argument values for each value of a chosen parameter (e.g. State) using the \textit{simple\_filepath} argument. These will override any fixed charge / marginal rate / PV argument values specified in the YML file. Any blank fields will be defaulted. File path is relative to the buildstock\_directory's \textit{resource} folder.  Note that currently only one parameter can be used as a dependency for the values which means that, for example, rates can vary by geography or by presence of a specific end use, but not both.
  %  \item Provide a tsv with an electricity tariff path for each option of a chosen parameter using the \textit{detailed\_filepath} argument. This otherwise works the same way as the second option.
%\end{itemize}
%For details on the emissions factors we typically use when running ResStock, please see \ref{utility bills}.

%\subsection{Simulation output report}
%What are the options and what does each yield.

%\subsection{Reporting measures}
%Options available, what does each produce, what do we commonly use in dataset publication?
%\subsection{Specifying upgrades}
%What is needed to specify an upgrade? Apply logic conventions. Use of elements &,*.
In ResStock, most of the details of upgrade specification occur directly in the project file under the \href{https://buildstockbatch.readthedocs.io/en/v2023.10.0/project_defn.html#upgrade-scenarios}{upgrades key}, using fields from the \textit{options\_lookup.tsv} file specified in logic blocks. Options specified for upgrades include which segment of the baseline the upgrade should be applied to, cost multipliers, the "reference" case which is important if doing a comparison against a business-as-usual scenario (especially for costs). If the upgrades section is not specified, only the building stock baseline will be simulated. Details of the upgrades associated with each ResStock data release can be found in the supporting upgrade measure documentation, for example, \cite{Present2024}.

%Each upgrade is specified using the following:
%\begin{itemize}
 %   \item upgrade\_name: (required) the name that will be in the outputs for this upgrade scenario
  %  \item options: a list of options to apply as part of this upgrade
   % \begin{itemize}
    %    \item option: (required) The option to apply, in the format parameter|option, which must match a parameter and option pair in the options\_lookup.tsv file in the repo
    %    \item apply\_logic: (optional) logic that defines which buildings to apply the upgrade to by using formal logic (if/and/or/not) on the baseline dwelling unit options
     %   \item costs: (optional) a list of upfront costs to implement the upgrade. Multiple costs can be entered and each is multiplied by a cost multiplier.
     %   \begin{itemize}
       %     \item value: a cost for the measure, which will be multiplied by the multiplier. For example, this could be a dollar per square foot of window area cost for a new window
       %     \item multiplier: the cost above is multiplied by this value, which is from a defined list of available cost multipliers made available in ResStock, most of which are calculated from the building energy model. Since there can be multiple costs, this permits both fixed and variable costs for upgrades that depend on the properties of the baseline building. For example, one cost multiplier is the total window area in square feet. Another is just the number one, denoted as "Fixed (1)", which is used for offsets or for costs that don't scale with baseline building parameters (e.g. we typically model costs for appliances like dishwashers as independent of the baseline building parameters)
       % \end{itemize}
       % \item  lifetime: (optional) lifetime in years of the upgrade. Useful for calculating total costs over a specified time horizon, assuming repeated replacement
    %\end{itemize}
    %\item package\_apply\_logic: (optional) the conditions under which this package of upgrades should be performed. Used in upgrade packages that have multiple options all of which should have the same apply logic. For example, if modeling a replacement of older windows with modern windows, the window is input as a separate option from the associated infiltration reduction but the two should apply to the exact same subset of models, so package\_apply\_logic would be useful
    %\item reference\_scenario: (optional) The upgrade\_name which should act as a reference to this upgrade to calculate savings. All this does is that reference\_scenario shows up as a column in the results csvs alongside the upgrade name; Buildstock Batch will not do the savings calculation but it can streamline downstream analysis.
%\end{itemize}



% There are a few exceptions to this, where the exact post-upgrade option does not have to exist in \textit{options\_lookup.tsv}, because the option row in options\_lookup.tsv instead provides arguments to an OpenStudio measure. However, the parameter|option pair must still be in \textit{options\_lookup.tsv}. An example of this is the Infiltration Reduction upgrade, which specifies a percentage infiltration reduction. The parameter|option pair might be Infiltration Reduction|20\%, which, if applied to a dwelling unit with a baseline infiltration of 20 ACH50, would result in a post-measure Infiltration Reduction of 16 ACH50, which is not in \textit{options\_lookup.tsv}.

ResStock upgrades are deterministic, not probabilistic, similar to how the baseline is constructed. You can specify, for example, that all dwelling units with a specific existing air conditioner in baseline get a specific new air conditioner in upgrade. Or you can use more complex logic and specify 10 different air conditioners in upgrade, based on any characteristic or combination of characteristics. But each dwelling unit will deterministically receive a specific new air conditioner based on the logic. This can cause challenges. One example is if specifying a new air conditioner for dwelling units that don't already have air conditioning, you might ideally specify a new, probabilistic range of cooling setpoints for those homes. However, this is not possible. This is why ResStock specifies cooling setpoints for every dwelling unit, whether or not the unit has air conditioning: so that if an upgrade run assigns air conditioning to that dwelling unit, the resulting setpoints are appropriately diverse. One can think of this situation as the dwelling unit's preference of a cooling setpoint if one had a cooling system. There are several other similar parameter option specifications in baseline that are not used to model the baseline but are available in case of certain upgrade option assignments. 

%%%% postprocessing

%%%Need to talk about failed simulations, how they get removed from SDR, and potential implications in representative count (why we say 550k but it's always less than that). How postprocessing sometimes also has to remove buildings with upgrades but no baseline.